\usepackage{palatino}   % Pretty Font
\usepackage{amsmath}    % American Mathematical Society (AMS) commands
\usepackage{amsthm}     % AMS Theorem commands
\usepackage{amssymb}    % AMS Symbols
\usepackage{semantic}   % Tools for typesetting PL semantics
\usepackage{fancyhdr}   % header/footer for each page
% \usepackage[margin=1in]{geometry}   % Give me reasonable margins
\usepackage{braket}     % Easy angle-bracket notation
\usepackage{mathpartir} % Used to typeset blocks of inference rules
%\usepackage[usenames,dvipsnames]{color}      % for colors
\usepackage[usenames]{color}      % for colors
\usepackage{rotating} % for sidewaysfigure
\usepackage{proof}
\usepackage{datetime}
\usepackage{pdflscape}
\usepackage{fancyvrb} % to use \verb in footnotes
\usepackage{enumitem} %% for resuming enumerations
\usepackage{stmaryrd}
\usepackage{natbib}

\usepackage{fullpage}
\usepackage{framed}
\usepackage{dsfont}
\usepackage{latexsym}
\usepackage{amsfonts}
\usepackage{mathrsfs}
\usepackage{aliascnt}
% \usepackage{pstricks}
% \usepackage{pst-all}
% \usepackage{pstricks-add}
% \usepackage{pst-plot}
\usepackage[title]{appendix}
\usepackage{graphicx}
% \usepackage{subfig}
% \usepackage{enumerate}
\PassOptionsToPackage{hyphens}{url}\usepackage[pdfpagelabels,pdfpagemode=None]{hyperref}

% \usepackage{algorithmic}
\usepackage{mathtools}
\usepackage{algorithm}
\usepackage{algpseudocode}
\usepackage[dvipsnames]{xcolor}
\usepackage{tikz}
\usepackage{subcaption}
\usepackage{bold-extra}

\usetikzlibrary{positioning,chains,fit,shapes,calc,arrows.meta,arrows,automata}


\newcommand{\rbsf}[1]{\textsf{\color{red}#1}}
\reservestyle{\errlang}{\rbsf}

\errlang{mismatch,underflow,unbound}


% Finite partial map
\newcommand{\finto}{\stackrel{\text{fin}}{\rightharpoonup}}
\newcommand{\pto}{\rightharpoonup}
\newcommand{\len}[1]{\lvert {#1} \rvert}

\theoremstyle{plain}
\newtheorem{theorem}{Theorem}
\newtheorem{lemma}{Lemma}
\newtheorem{proposition}{Proposition}
\newtheorem{conjecture}{Conjecture}
\newtheorem{corollary}{Corollary}
\newtheorem{definition}{Definition}
\newtheorem{problem}{Problem}

\theoremstyle{definition}
\newtheorem{example}{Example}

% When a property is true by the Induction Hypothesis...
\newcommand{\ih}[1]{\stackrel{\text{I.H.}}{#1}}


\theoremstyle{remark}
\newtheorem*{case}{Case}
\newtheorem*{note}{Note}
\newtheorem*{notation}{Notation}
\newtheorem*{solution}{Solution}
\newenvironment{subcase}[1][]{
  \mbox{}\\\mbox{}\quad
  \begin{minipage}{1.0\linewidth}
    \begin{case}[#1]
      }{
    \end{case}
  \end{minipage}
}

% The following helps fix having too much vertical space when writing
% predicates or functions applied to derivation trees
\newcommand{\sarray}[1]{
  {\begin{array}[c]{@{}l@{}} #1
      \end{array}}}

\newcommand{\fdeduce}[2]{\ensuremath{#2 :: #1}}

%% Random math macros

% Set Theory
\newcommand{\tuple}[1]{\braket{#1}}
\newcommand{\setint}{\ensuremath{\cap}}
\newcommand{\setdiff}{\mathrel{\backslash}}
\newcommand{\union}{\ensuremath{\cup}}   % the set union symbol
\newcommand{\Z}{\ensuremath{\mathbb{Z}}} % the set of integers
\newcommand{\N}{\ensuremath{\mathbb{N}}} % the set of naturals
\newcommand{\B}{\ensuremath{\mathbb{B}}} % the set of booleans
\newcommand{\R}{\ensuremath{\mathbb{R}}}
\newcommand{\Pow}{\ensuremath{\mathcal{P}}} % The powerset

% Logic
%\renewcommand{\land}{\mathop{{\wedge}}}
\newcommand{\limplies}{\mathop{{=>}}}
\newcommand{\ltrue}{{\top}}
%\renewcommand{\lor}{\mathop{{\vee}}}
\newcommand{\lfalse}{{\bot}}
\newcommand{\lequiv}{\mathop{{\equiv}}}



%% Comments
\newcommand{\comment}[2]{{\color{red}[{\sc #1:} \textsf{#2}]}}
%\renewcommand{\comment}[2]{}
\newcommand{\ron}[1]{\comment{RG}{#1}}


%%
%% Macros for syntax definitions
%%
\newcommand{\mvar}[1]{\ensuremath{\mathit{#1}}}
\newcommand{\seq}[1]{\ensuremath{\overline{#1}}}
\newcommand{\hole}{\square}
\newcommand{\nsubst}[3]{{[#2 \Mapsto #1]#3}}
\newcommand{\ansubst}[3]{{\underline{[#2 \Mapsto #1]}#3}}

%ron Redefine \subst
\newcommand{\subst}[3]{\ensuremath{[#1/#2]#3}}
\newcommand{\asubst}[3]{\ensuremath{\underline{[#1/#2]}#3}}
\newcommand{\remap}[3]{\ensuremath{#1[#2 \mapsto #3]}}


%%
%% inferbox: an environment for typesetting a group of inference rules
%%
\newenvironment{inferbox}[1][\textwidth]
{\begin{minipage}{#1}\begin{mathpar}}
      {\end{mathpar}\end{minipage}}


%% buinfer: build an inference tree bottom up, rather than top-down.  Sometimes
%% this is easier to do, especially when building complicated trees

\newcommand{\binference}[3][]{\inference[#1]{#3}{#2}}
%\newcommand{\binference}[3][]{\infer[\text{#1}]{#2}{#3}}

% silent inference - ignore the labels
\newcommand{\sinference}[3][]{\inference[]{#2}{#3}}

% Derivation-related macros
\newcommand{\D}{\mathcal{D}}
\newcommand{\E}{\mathcal{E}}
\newcommand{\F}{\mathcal{F}}
\newcommand{\ddd}{\raisebox{0.2em}[1.1em]{$\vdots$}}


%% block - used to indent code or create newlines of math stuff.
\newenvironment{block}[1][t]
{\begin{array}[#1]{@{}l@{}}}
    {\end{array}}

% Block, or any array[t]/array[c]/array[b] does not play well with 
% stretchy parens or brackets.  Use this instead.
% It protects everything in another array
\newenvironment{roundbrack}{\left(\begin{array}{@{}l@{}}}
    {\end{array}\right)}


% mark something in gray outline
\definecolor{lightgray}{gray}{0.80}
\newcommand{\Gbox}[1]{\colorbox{lightgray}{$#1$}}

\newcommand{\gbox}[1]{{\setlength{\fboxsep}{0pt}\colorbox{lightgray}{#1}}}

%% Math ligatures (thanks to the semantic package) that make it
%% easier to typeset math using readable LaTeX text.
\mathlig{|-->}{\longmapsto}
\mathlig{[]}{\square}
\mathlig{||}{\mathrel{\color{black}\Downarrow}}
\mathlig{!!}{{\color{blue}!}}
\mathlig{;;}{\mbsf{;}}
\mathlig{::==}{\mathbin{\color{blue}\texttt{\bf \upshape :=}}}
%\mathlig{::==}{\mbsf{:=}}
\mathlig{|}{\mid}
\mathlig{..}{{\color{blue}.}}
\mathlig{@-}{\mathbin{\color{blue}\texttt{\bf \upshape -}}}
\mathlig{++}{\mathbin{\color{blue}\texttt{\bf \upshape +}}}
\mathlig{**}{\mathbin{\color{blue}\mathbf{*}}}
\mathlig{==}{\mathbin{\color{blue}\texttt{\bf \upshape =}}}
\mathlig{,,}{{\color{blue},}}
\mathlig{[[}{\mbsf{[}}
\mathlig{]]}{\mbsf{]}}
\mathlig{@:}{\mbsf{:}}
\mathlig{@(}{{\color{blue}\texttt{\bf \upshape (}}}
\mathlig{@)}{{\color{blue}\texttt{\bf \upshape )}}}
%\mathlig{@(}{\mbsf{(}}
%\mathlig{)@}{\mbsf{)}}
\newcommand{\tbrack}[1]{{\triangleleft#1\triangleright}}
\newcommand{\thunk}[1]{\tbrack{\hspace{2pt}#1\hspace{2pt}}}
\newcommand{\closure}{\texttt{\textbf{\#<closure>}}}

\newcommand{\unit}{{\color{blue}\texttt{\bf ()}}}


% Use smallcaps for the names of object language sets.
\newcommand{\oblset}[1]{\textsc{#1}}

% Some common Sets
\newcommand{\TreeSet}{\oblset{Tree}}
\newcommand{\Tree}{\oblset{Tree}}
\newcommand{\Deriv}{\oblset{Deriv}}
\newcommand{\Term}{\oblset{Term}}
\newcommand{\Value}{\oblset{Value}}
\newcommand{\Var}{\oblset{Var}}
\newcommand{\Pgm}{\oblset{Pgm}}
\newcommand{\Obs}{\oblset{Obs}}
\newcommand{\Frame}{\oblset{Frame}}
\newcommand{\ECtxt}{\oblset{ECtxt}}
\newcommand{\Ctxt}{\oblset{Ctxt}}
\newcommand{\Redex}{\oblset{Redex}}
\newcommand{\Type}{\oblset{Type}}

% Some custom notations for the metalanguage
\reservestyle{\mtlang}{\textit}
\renewcommand{\eval}{\mathit{eval}}
\newcommand{\dom}{\mathit{dom}}
\newcommand{\cod}{\mathit{cod}}
\newcommand{\unload}{\mathit{unload}}

% Some custom notations for object language stuff
% Typeset object language notation in blue with sans serif font
\newcommand{\blue}[1]{{{\color{blue}#1}}}
\newcommand{\tbsf}[1]{\textsf{\color{blue}#1}}
\newcommand{\mbsf}[1]{\mathsf{\color{blue}#1}}
\newcommand{\ol}[1]{\ensuremath{\mbsf{#1}}}
\reservestyle{\oblang}{\tbsf}



%
% Languages:
%

% Logic
% RG: This is actually metalanguage:  there is a logic object language which 
% should go here.  Move Logic up

% Formalized
\newcommand{\band}{\mathop{\mbsf{\wedge}}}
\newcommand{\bimplies}{\mathop{\mbsf{=>}}}
\newcommand{\btrue}{\mbsf{\top}}
\newcommand{\bor}{\mathop{\mbsf{\vee}}}
\newcommand{\bnot}{\mathop{\mbsf{\lnot}}}
\newcommand{\bfalse}{\mbsf{\bot}}
\newcommand{\bequiv}{\mathop{\mbsf{\equiv}}}

\newcommand{\jtrue}{\textbf{true}}



% Booleans
\newcommand{\Bool}{\oblset{Bool}}
\oblang{if-then-else[if],true,false,if[if\;],then[\;then\;],else[\;else\;]
}


% Arithmetic
% RG: Why do I have both Num and Nat?
\newcommand{\Num}{\oblset{Num}}
\oblang{z,succ,pred,zero?}
\newcommand{\nv}{\mathit{nv}}


% IMP <: Booleans
\newcommand{\AExp}{\oblset{AExp}}
\newcommand{\BExp}{\oblset{BExp}}
\newcommand{\Com}{\oblset{Com}}
\newcommand{\Env}{\oblset{Env}}
\newcommand{\DO}{\oblset{DO}}
\newcommand{\Store}{\oblset{Store}}
\newcommand{\SO}{\oblset{SO}}
\newcommand{\Loc}{\oblset{Loc}}
\newcommand{\PLOC}{\Pow(\Loc)}
\newcommand{\ACfg}{\oblset{ACfg}}
\newcommand{\BCfg}{\oblset{BCfg}}
\newcommand{\CCfg}{\oblset{CCfg}}

\oblang{while[while\;],do[\;do\;],skip}

\newcommand{\X}{\mbsf{X}}
\newcommand{\Y}{\mbsf{Y}}
\newcommand{\ZZ}{\mbsf{Z}}

\newcommand{\bv}{\mathit{bv}}
\newcommand{\bsa}{\Downarrow_{\AExp}}
\newcommand{\bsb}{\Downarrow_{\BExp}}
\newcommand{\bsc}{\Downarrow_{\Com}}

\newenvironment{whileblock}[2][t]
{\begin{block}[#1]\<while> \mbsf{#2} \<do>\\
    \quad\begin{block} \color{blue}}
      {\end{block}\end{block}}


% Doesn't work because of the end-environment part (grr...)
% \newenvironment{doblock}[2][t]
%   {\begin{block}[#1]\<do>\\
%       \quad\begin{block} \color{blue}}
%     {\end{block} \<while> \mbsf{#2} \end{block}}



% Procedures (Lambda Calculus)
\newcommand{\blambda}{\ensuremath{{\color{blue}\lambda}}}
\newcommand{\FV}{\mathit{FV}}
\newcommand{\BV}{\mathit{BV}}
\newcommand{\Vars}{\mathit{Vars}}
\newcommand{\Locs}{\mathit{Locs}}
\oblang{var,apply,procedure}
\oblang{tlet[let],let[let\;],dlet[dlet\;],in[\;in\;],nin[in\;]}

\newcommand{\w}{\mbsf{w}}
\newcommand{\x}{\mbsf{x}}
\newcommand{\y}{\mbsf{y}}
\newcommand{\z}{\mbsf{z}}

\newenvironment{letblock}[2][t]
{\begin{block}[#1]\<let> \mbsf{#2} \\
    \<nin>\begin{block} \color{blue}}
      {\end{block}\end{block}}


% Recursion
\oblang{fix[fix\;],rec[rec\;]}

\newenvironment{fixblock}[2][t]
{\begin{block}[#1]\<fix> \mbsf{#2}. \\
    \quad\begin{block} \color{blue}}
      {\end{block}\end{block}}


% Unit
\oblang{unit,Unit}

% References
\oblang{ref[ref\;]}


% Exceptions
\oblang{raise[raise\;],try[try\;],handle[\;handle\;]}

% Continations
% use \braket{E} for continuation values
\oblang{letcc[let/cc\;],throw[throw\;]}

\newenvironment{letccblock}[2][t]
{\begin{block}[#1]\<letcc> \mbsf{#2} \\
    \<nin>\begin{block} \color{blue}}
      {\end{block}\end{block}}


% Pairs
\newcommand{\pr}[2]{\mbsf{(}#1\mbsf{,\,}#2\mbsf{)}}
\oblang{fst[fst\;],snd[snd\;]}


% Sums
\oblang{inr,inl,case[case\;],of[\;of]}

\newcommand{\tcase}[5]{
  \begin{block}
    \<case> #1 \<of> \\
    \quad
    \begin{block}
      \<inl>\;#2  .. #3 \\
      \<inr>\;#4  .. #5 \\
    \end{block}
  \end{block}
}


% Recursive Types
\oblang{fold,unfold}

% Errors
\oblang{error}



%
% Other stuff
%

% CPS translation function
\newcommand{\cps}[1]{|[ #1 |]}


%%% Local Variables: 
%%% mode: latex
%%% TeX-master: t
%%% End: 

% Algorithms
\newcommand{\matroid}{\mathcal{M}}
\newcommand{\classp}{\mathsf{P}}
\newcommand{\classnp}{\mathsf{NP}}
\newcommand{\classnpo}{\mathsf{NPO}}
\newcommand{\yes}{\mathsf{YES}}
\newcommand{\no}{\mathsf{NO}}
\newcommand{\preduce}{\leq_{\classp}}
\newcommand{\ALGO}{\mathcal{A}}
\newcommand{\alg}{\mathsf{ALG}}
\newcommand{\opt}{\mathsf{OPT}}
\newcommand{\eps}{\epsilon}
\newcommand{\equal}{\texttt{\,=\,}}
\let\oldemptyset\emptyset
\let\emptyset\varnothing

\algrenewcommand\algorithmicrequire{Input:}
\algrenewcommand\algorithmicensure{Output:}
\algdef{SE}[SUBALG]{Indent}{EndIndent}{}{\algorithmicend\ }%
\algtext*{Indent}
\algtext*{EndIndent}

\renewcommand\proofname{\textsc{Proof.}}